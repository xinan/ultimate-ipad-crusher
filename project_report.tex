\documentclass[12pt,a4paper,oneside]{report}

\usepackage{enumitem}
\usepackage{setspace}
\usepackage[none]{hyphenat}
\usepackage{algorithm} 
\usepackage{algorithmic}
\usepackage{amsmath}

\begin{document}

%===============================================================================
% Start of Cover Page
%===============================================================================
\title{\Huge\textbf{The iPad Crusher \vfill
	\large Project Report \\ \bigskip
	for \\ \bigskip
	CS2309 CS Research Methodology \\ \bigskip
	\large AY15/16 Sem 1 \vfill}
	}
\author{
	Department of Computer Science \\\\
	School of Computing \\\\
	National University of Singapore}
%\date{}

\maketitle
%===============================================================================
% End of Cover Page
%===============================================================================

\pagenumbering{roman}

%===============================================================================
% Start of Title Page
%===============================================================================
\begin{titlepage}
\addcontentsline{toc}{chapter}{Title Page}
\thispagestyle{plain}
\begin{center}
\textbf{\Huge The iPad Crusher \vfill
	\large Project Report \\ \bigskip
	for \\ \bigskip
	CS2309 CS Research Methodology \\ \bigskip
	AY15/16 Sem 1} \vfill \vfill
\large Members: \\ \bigskip
\begin{minipage}[t]{0.3\textwidth}
\begin{flushleft}
Lai Hoang Dung \\ \bigskip
Lim Kiat \\ \bigskip
Liu Xinan \\ \bigskip
Tran Tien Dat
\end{flushleft}
\end{minipage}
\begin{minipage}[t]{0.3\textwidth}
\begin{flushright}
A0131125Y \\ \bigskip
A0123456X \\ \bigskip
A0130195M \\ \bigskip
A0131140E
\end{flushright}
\end{minipage} \vfill
Supervisor: \\ \bigskip
Prof. Sim Khe Chai
\end{center}
\end{titlepage}
%===============================================================================
% End of Title Page
%===============================================================================

%===============================================================================
% Start of Abstract
%===============================================================================
\begin{abstract}
\setcounter{page}{2}
\thispagestyle{plain}
\onehalfspacing
\addcontentsline{toc}{chapter}{Abstract}
In this project, we were asked to design the game logic for ``iPad Crusher'', an iOS game that Prof. Sim is developing as an adaptation to the Egg Dropping Problem. Given a certain number of iPads and strength levels, the player is asked to find the lowest strength level that would break an iPad. Similar to the Egg Dropping Puzzle, the goal of the game is to find the best strategy so that it involves the fewest number of crushing in the worst-case scenario. \\\\
In this report we present two approaches in designing the game logic, the first is to generate all the decision trees and design the game logic according to the decision trees. The other one is to just generate the cost table and make decision based on the costs. We would discuss the advantages and disadvantages of each of the approaches, as well as different implementations of them. Lastly we would discuss some interesting extension of the problem such as adding an extra cost for breaking an iPad, and a variant of it with unlimited number of iPads, etc. In the end we conclude that for the original game logic, the cost table approach would be the better choice, whereas for some extensions, the decision tree approach would be better.

\subsubsection*{Subject Descriptors:\\}
\begin{description}[labelindent=1cm]
	\item[I.2.8] Dynamic Programming
	\item[K.8.0] Games
	\item[G.2.1] Combinatorics
\end{description}

\subsubsection{Keywords:}
\begin{description}[labelindent=1cm]
	\item Game Logic, Egg Dropping, Dynamic Programming, Combinatorics
\end{description}

\end{abstract}
%===============================================================================
% End of Abstract
%===============================================================================

%===============================================================================
% Start of Acknowledgements
%===============================================================================
\renewcommand{\abstractname}{Acknowledgements}
\begin{abstract}
\setcounter{page}{3}
\thispagestyle{plain}
\onehalfspacing
\addcontentsline{toc}{chapter}{Acknowledgements}
We could like to thank Prof. Sim Khe Chai for the patience and guidance along the semester. Without Prof. Sim, we would not have gone this far. I don't know what to write here you guys please fill in. \\\\
(TODO)
\end{abstract}
%===============================================================================
% End of Acknowledgements
%===============================================================================

\onehalfspacing
\tableofcontents
\thispagestyle{empty}
\addtocontents{toc}{\protect\thispagestyle{empty}}
\pagenumbering{arabic}

%===============================================================================
% Start of Chapter 1
%===============================================================================
\chapter{Introduction}
\setcounter{page}{1}
\section{About the project}
Lorem ipsum dolor sit amet, consectetur adipiscing elit. Duis quis lacus posuere, tincidunt mauris vitae, bibendum sapien. Suspendisse a eros pretium, pretium libero eget, ornare ante. Donec efficitur gravida pellentesque. Curabitur sit amet pulvinar tortor. In eu risus commodo, elementum tellus ac, commodo mauris. Vestibulum ante ipsum primis in faucibus orci luctus et ultrices posuere cubilia Curae; Integer hendrerit arcu sem, convallis dignissim nisi faucibus at. Praesent ut rutrum libero, sollicitudin blandit dui. Donec mi nisi, cursus quis accumsan et, molestie ut nisi. Sed egestas enim in dolor tincidunt dictum. Sed eu semper sem. Sed elementum augue ac pretium eleifend. Ut massa tortor, rhoncus a mi ac, congue ullamcorper eros. Maecenas imperdiet orci quis dui rhoncus, id rhoncus mi condimentum. Nunc blandit egestas leo, sit amet porttitor mi malesuada nec. Pellentesque id bibendum purus, a porta risus.

%===============================================================================
% End of Chapter 1
%===============================================================================

%===============================================================================
% Start of Chapter 2
%===============================================================================
\chapter{Existing work}
I put random chapter titles here. Feel free to change. And this is how you cite a source \cite{randomblogpost}.

\section{What does the fox say?}
\subsection{Introduction}
Lorem ipsum dolor sit amet, consectetur adipiscing elit. Duis quis lacus posuere, tincidunt mauris vitae, bibendum sapien. Suspendisse a eros pretium, pretium libero eget, ornare ante. Donec efficitur gravida pellentesque. Curabitur sit amet pulvinar tortor. In eu risus commodo, elementum tellus ac, commodo mauris. Vestibulum ante ipsum primis in faucibus orci luctus et ultrices posuere cubilia Curae; Integer hendrerit arcu sem, convallis dignissim nisi faucibus at. Praesent ut rutrum libero, sollicitudin blandit dui. Donec mi nisi, cursus quis accumsan et, molestie ut nisi. Sed egestas enim in dolor tincidunt dictum. Sed eu semper sem. Sed elementum augue ac pretium eleifend. Ut massa tortor, rhoncus a mi ac, congue ullamcorper eros. Maecenas imperdiet orci quis dui rhoncus, id rhoncus mi condimentum. Nunc blandit egestas leo, sit amet porttitor mi malesuada nec. Pellentesque id bibendum purus, a porta risus. \\\\

%===============================================================================
% End of Chapter 2
%===============================================================================

%===============================================================================
% Start of Chapter 3
%===============================================================================
\chapter{The decision tree approach}
A short brief description of the approach here.

\section{A brute force algorithm}
Lorem ipsum dolor sit amet, consectetur adipiscing elit. Duis quis lacus posuere, tincidunt mauris vitae, bibendum sapien. Suspendisse a eros pretium, pretium libero eget, ornare ante. Donec efficitur gravida pellentesque. Curabitur sit amet pulvinar tortor. In eu risus commodo, elementum tellus ac, commodo mauris. Vestibulum ante ipsum primis in faucibus orci luctus et ultrices posuere cubilia Curae; Integer hendrerit arcu sem, convallis dignissim nisi faucibus at. Praesent ut rutrum libero, sollicitudin blandit dui. Donec mi nisi, cursus quis accumsan et, molestie ut nisi. Sed egestas enim in dolor tincidunt dictum. Sed eu semper sem. Sed elementum augue ac pretium eleifend. Ut massa tortor, rhoncus a mi ac, congue ullamcorper eros. Maecenas imperdiet orci quis dui rhoncus, id rhoncus mi condimentum. Nunc blandit egestas leo, sit amet porttitor mi malesuada nec. Pellentesque id bibendum purus, a porta risus.

\section{A dynamic programming algorithm}

\section{Pros and Cons}

%===============================================================================
% End of Chapter 3
%===============================================================================

%===============================================================================
% Start of Chapter 4
%===============================================================================
\chapter{The cost table appraoch}
A short brief description of the approach here.

\section{A brute force algorithm}
Lorem ipsum dolor sit amet, consectetur adipiscing elit. Duis quis lacus posuere, tincidunt mauris vitae, bibendum sapien. Suspendisse a eros pretium, pretium libero eget, ornare ante. Donec efficitur gravida pellentesque. Curabitur sit amet pulvinar tortor. In eu risus commodo, elementum tellus ac, commodo mauris. Vestibulum ante ipsum primis in faucibus orci luctus et ultrices posuere cubilia Curae; Integer hendrerit arcu sem, convallis dignissim nisi faucibus at. Praesent ut rutrum libero, sollicitudin blandit dui. Donec mi nisi, cursus quis accumsan et, molestie ut nisi. Sed egestas enim in dolor tincidunt dictum. Sed eu semper sem. Sed elementum augue ac pretium eleifend. Ut massa tortor, rhoncus a mi ac, congue ullamcorper eros. Maecenas imperdiet orci quis dui rhoncus, id rhoncus mi condimentum. Nunc blandit egestas leo, sit amet porttitor mi malesuada nec. Pellentesque id bibendum purus, a porta risus.

\section{A dynamic programming algorithm}

\section{Pros and Cons}

%===============================================================================
% End of Chapter 4
%===============================================================================

%===============================================================================
% Start of Chapter 5
%===============================================================================
\chapter{The online combinatorics appraoch}
A short brief description of the approach here.

\section{Correctness}
Lorem ipsum dolor sit amet, consectetur adipiscing elit. Duis quis lacus posuere, tincidunt mauris vitae, bibendum sapien. Suspendisse a eros pretium, pretium libero eget, ornare ante. Donec efficitur gravida pellentesque. Curabitur sit amet pulvinar tortor. In eu risus commodo, elementum tellus ac, commodo mauris. Vestibulum ante ipsum primis in faucibus orci luctus et ultrices posuere cubilia Curae; Integer hendrerit arcu sem, convallis dignissim nisi faucibus at. Praesent ut rutrum libero, sollicitudin blandit dui. Donec mi nisi, cursus quis accumsan et, molestie ut nisi. Sed egestas enim in dolor tincidunt dictum. Sed eu semper sem. Sed elementum augue ac pretium eleifend. Ut massa tortor, rhoncus a mi ac, congue ullamcorper eros. Maecenas imperdiet orci quis dui rhoncus, id rhoncus mi condimentum. Nunc blandit egestas leo, sit amet porttitor mi malesuada nec. Pellentesque id bibendum purus, a porta risus.

\section{Pros and Cons}

%===============================================================================
% End of Chapter 5
%===============================================================================

%===============================================================================
% Start of Chapter 6
%===============================================================================
\chapter{Interesting extensions}
\section{iPad with fixed cost}
What happens if we have to pay a penalty for each crushed iPad. How would it affect the game logic?
To tackle this extension, we realise we can re-use our cost table approach, but modify the comparison step slightly to also include the fixed cost. 
Using the table cost approach, we would calculate the cost of the whole DP table during the game initialisation, and subsequently use the value already calculated in the table cell to make the game decisions
The pseudo code to calculate the DP table is as follows
\begin{algorithm}
        \caption{Calculate the cost table for fixed iPad cost}
        \begin{algorithmic}[1]
            \REQUIRE N, L, iPadCost
            \STATE Initialise a 2D table DP[N+1][L+1]
            \STATE For all table cells (n, l) where n = 1, initialise the value of the cell to be 1 + iPadCost
            \STATE For all table cells (n, l) where L = 1, initialise the value of the cell to be 0
            \STATE For all other tables cell (n, l) , the value at that cell is equal to min(1 + max(DP[x-1][n-1] + iPadCost, DP[L-x][n])) for $1 \leq x \leq l$
        \end{algorithmic}
\end{algorithm}
We can see that the difference between this and the previous cost table approach without iPad cost is only that the iPad cost is also added to the original cost in each cell
\section{iPad with dynamic cost}
Let's make the problem even more complex. How do we tackle the problem if the cost of the iPad is not fixed, but changing depending on how many iPad has been crushed. 
\\\\We can first define a function f, where f(n) denotes the cost of the $n^{th}$ being dropped
Intuitively, we can see that this problem is very close to the fixed iPad cost problem. We can again use the cost table approach to pre-calculate the DP cost table. The difference lies in the fact that we also now need to consider how many iPads have been crushed before.
Therefore, to extend the solution for the fixed cost problem to also work in this, we can also store an attribute denoting the number of Ipad crushed.
The pseudo code to calculate the DP table is as follows
\begin{algorithm}
        \caption{Calculate the cost table for fixed iPad cost}
        \begin{algorithmic}[1]
            \REQUIRE N, L, iPadCost
            \STATE Initialise a 2D table DP[N+1][L+1]
            \STATE For all table cells (n, l) where n = 1, initialise the value of the cell to be 1 + iPadCost
            \STATE For all table cells (n, l) where L = 1, initialise the value of the cell to be 0
            \STATE For all other tables cell (n, l) , the value at that cell is equal to min(1 + max(DP[x-1][n-1] + iPadCost, DP[L-x][n])) for $1 \leq x \leq l$
        \end{algorithmic}
\end{algorithm}
%===============================================================================
% End of Chapter 6
%===============================================================================

%===============================================================================
% Start of Chapter 7
%===============================================================================
\chapter{Conclusion}
Lorem ipsum dolor sit amet, consectetur adipiscing elit. Duis quis lacus posuere, tincidunt mauris vitae, bibendum sapien. Suspendisse a eros pretium, pretium libero eget, ornare ante. Donec efficitur gravida pellentesque. Curabitur sit amet pulvinar tortor. In eu risus commodo, elementum tellus ac, commodo mauris. Vestibulum ante ipsum primis in faucibus orci luctus et ultrices posuere cubilia Curae; Integer hendrerit arcu sem, convallis dignissim nisi faucibus at. Praesent ut rutrum libero, sollicitudin blandit dui. Donec mi nisi, cursus quis accumsan et, molestie ut nisi. Sed egestas enim in dolor tincidunt dictum. Sed eu semper sem. Sed elementum augue ac pretium eleifend. Ut massa tortor, rhoncus a mi ac, congue ullamcorper eros. Maecenas imperdiet orci quis dui rhoncus, id rhoncus mi condimentum. Nunc blandit egestas leo, sit amet porttitor mi malesuada nec. Pellentesque id bibendum purus, a porta risus.

%===============================================================================
% End of Chapter 7
%===============================================================================

%===============================================================================
% Start of References
%===============================================================================
\renewcommand{\bibname}{References}
\begin{thebibliography}{1} \addcontentsline{toc}{chapter}{References}
	\bibitem{randomblogpost} Egg Drop Puzzle. http://michaelbrundage.com/note/2014/04/26/egg-drop/
\end{thebibliography}

%===============================================================================
% End of References
%===============================================================================
\end{document}
